\section{Matrizen}


\begin{question}
 Finden Sie eine Matrix $T \in \Mat_4(k)$, so dass für alle $A \in \Mat_n(k)$ das Matrixprodukt $A \cdot T$ durch vertauschen der ersten und dritten Spalte von $A$ entsteht. Ist $T$ eindeutig?
\end{question}
\begin{solution}
 Falls eine solche Matrix $T$ existiert muss
 \[
  T = I \cdot T =
  \begin{pmatrix}
   0 & 0 & 1 & 0 \\
   0 & 1 & 0 & 0 \\
   1 & 0 & 0 & 0 \\
   0 & 0 & 0 & 1
  \end{pmatrix}.
 \]
 Das zeigt die Eindeutigkeit und wie $T$ aussehen muss. Für alle
 \[
  A =
  \begin{pmatrix}
   a_{11} & a_{12} & a_{13} & a_{14} \\
   a_{21} & a_{22} & a_{23} & a_{24} \\
   a_{31} & a_{32} & a_{33} & a_{34} \\
   a_{41} & a_{42} & a_{43} & a_{44}
  \end{pmatrix}
 \]
 ist
 \[
  A \cdot T =
  \begin{pmatrix}
   a_{11} & a_{12} & a_{13} & a_{14} \\
   a_{21} & a_{22} & a_{23} & a_{24} \\
   a_{31} & a_{32} & a_{33} & a_{34} \\
   a_{41} & a_{42} & a_{43} & a_{44}
  \end{pmatrix}
  \cdot
  \begin{pmatrix}
   0 & 0 & 1 & 0 \\
   0 & 1 & 0 & 0 \\
   1 & 0 & 0 & 0 \\
   0 & 0 & 0 & 1
  \end{pmatrix}
  =
  \begin{pmatrix}
   a_{13} & a_{12} & a_{11} & a_{14} \\
   a_{23} & a_{22} & a_{21} & a_{24} \\ 
   a_{33} & a_{32} & a_{31} & a_{34} \\
   a_{43} & a_{42} & a_{41} & a_{44}.
  \end{pmatrix},
 \]
 also erfüllt $T$ die gewünschte Eigenschaft.
\end{solution}




% TODO: Eigenschaften des Matrizenringes.


\begin{question}
 Für $A \in \Mat_n(k)$ sei
 \[
  L_A \colon \Mat_n(k) \rightarrow \Mat_n(k), B \mapsto A \cdot B.
 \]
 die Linksmultiplikation mit $A$. Zeigen Sie:
 \begin{enumerate}
  \item
   Für alle $A \in \Mat_n(k)$ ist $L_A$ eine lineare Abbildung.
  \item
   Die Abbildung
   \[
    \Phi \colon \Mat_n(k) \rightarrow \End(\Mat_n(k)), A \mapsto L_A
   \]
   ist linear.
  \item
   $\Phi$ ist injektiv.
 \end{enumerate}
\end{question}
\begin{solution}
 \begin{enumerate}
  \item
   Da für alle $B_1, B_2 \in \Mat_n(k)$
   \[
    L_A(B_1 + B_2) = A (B_1 + B_2) = A B_1 + A B_2 = L_A(B_1) + L_A(B_2)
   \]
   ist $L_A$ additiv. Da für alle $\lambda \in k$, $B \in \Mat_n(k)$
   \[
    L_A(\lambda B) = A(\lambda B) = \lambda A B = \lambda L_A(B)
   \]
   ist $L_A$ mit der Skalarmultiplikation verträglich. Also ist $L_A$ linear.
  \item
   Für alle $A_1, A_2 \in \Mat_n(k)$ ist für alle $B \in \Mat_n(k)$
   \begin{align*}
    L_{A_1 + A_2}(B)
    &= (A_1 + A_2) B
    = A_1 B + A_2 B \\
    &= L_{A_1}(B) + L_{A_2}(B)
    = (L_{A_1} + L_{A_2})(B),
   \end{align*}
   und somit
   \[
    \Phi(A_1 + A_2) = L_{A_1 + A_2} = L_{A_1} + L_{A_2} = \Phi(A_1) + \Phi(A_2).
   \]
   Also ist $\Phi$ additiv.
   
   Für alle $\lambda \in k$, $A \in \Mat_n(k)$ ist für alle $B \in \Mat_n(k)$
   \[
    L_{\lambda A}(B) = (\lambda A) B = \lambda A B = \lambda L_A(B) = (\lambda L_A)(B)
   \]
   und somit
   \[
    \Phi(\lambda A) = L_{\lambda A} = \lambda L_A = \lambda \Phi(A).
   \]
   Also ist $\Phi$ mit der Skalarmultiplikation verträglich.
  \item
   Für $A, A' \in \Mat_n(k)$ mit $\Phi(A) = \Phi(A')$ ist
   \[
    A = A \cdot I = L_A(I) = \Phi(A)(I) = \Phi(A')(I) = L_{A'}(I) = A' \cdot I = A'.
   \]
   Also ist $\Phi$ injektiv.
 \end{enumerate}
\end{solution}


\begin{question}
 Es seien $A_1, \dotsc, A_r \in \Mat_n(k)$ und $A_1 \dotsm A_r$ sei invertierbar. Zeigen Sie, dass dann $A_i$ für alle $i=1, \dotsc, r$ invertierbar ist.
\end{question}
\begin{solution}
 Die Aussage lässt sich per Induktion über $r$ beweisen: Für $r = 1$ ist die Aussage klar.
 
 Angenommen, die Aussage gilt für $r \geq 1$, und es seien $A_1, \dotsc, A_{r+1} \in \Mat_n(k)$, so dass $S \coloneqq A_1 \dotsm A_{r+1}$ invertierbar ist. Für $x \in k^n$ mit $A_{r+1} x = 0$ ist auch
 \[
  S x = A_1 \dotsm A_{r+1} x = 0.
 \]
 Also ist
 \[
  \ker S \supseteq \ker A_{r+1}.
 \]
 Da $S$ invertierbar ist, ist $\ker S = 0$. Also ist auch $\ker A_{r+1} = 0$ und $A$ somit invertierbar.
 
 Da $A_{r+1}$ invertierbar ist, lässt sich $A_1 \dotsm A_{r+1} = S$ durch Multiplikation mit $A_{r+1}^{-1}$ von rechts zu
 \[
  A_1 \dotsm A_r = S A_{r+1}^{-1} \eqqcolon T
 \]
 umformen. Da $S$ und $A_{r+1}^{-1}$ invertierbar sind, ist auch $T$ invertierbar. Nach Induktionsvoraussetzung sind daher auch $A_i$ für alle $i = 1, \dotsc, r$ invertierbar. Zusammen mit der Invertierbarkeit von $A_{r+1}$ ist als $A_i$ für alle $i = 1, \dotsc, r+1$ invertierbar.
\end{solution}


\begin{question}
 Es sei
 \[
  A =
  \begin{pmatrix*}[r]
    2 & -4 &  6 &  4 & -2 \\
   -1 &  2 & -3 & -2 &  1 \\
   -3 &  6 & -9 & -6 &  3 \\
    1 & -2 &  3 &  2 & -1 \\
   -2 &  4 & -6 & -4 &  2
  \end{pmatrix*}
  \in \Mat(5 \times 5, \R).
 \]
 \begin{enumerate}
  \item
   Finden Sie Matrizen $B \in \Mat(5 \times 1, \R)$ und $C \in \Mat(1 \times 5, \R)$ mit $A = BC$.
  \item
   Berechnen Sie $A^{2016}$.
 \end{enumerate}
\end{question}
\begin{solution}
 \begin{enumerate}
  \item
   Alle Spalten und Zeilen von $A$ sind Vielfache voneinander. Durch geschicktes Hinsehen ergibt sich für
   \[
    B \coloneqq
   \begin{pmatrix}
      2 \\
     -1 \\
     -3 \\
      1 \\
     -2
    \end{pmatrix} \in \Mat(5 \times 1, \R)
    \quad \text{und} \quad
    C \coloneqq
    \begin{pmatrix}
     1 & -2 & 3 & 2 & -1
    \end{pmatrix} \in \Mat(1 \times 5, \R),
   \]
   dass $A = BC$.
  \item
   Es ist
   \[
    CB = \begin{pmatrix}-1\end{pmatrix} \in \Mat_1(\R).
   \]
   Deshalb ist
   \[
    A^{2016} = (BC)^{2016} = B (CB)^{2015} C = B \begin{pmatrix}-1\end{pmatrix}^{2015} C = -BC = -A.
   \]
   Allgemeiner ist für alle $n \geq 1$
   \[
    A^n =
    \begin{cases}
      A & \text{falls $n$ ungerade ist}, \\
     -A & \text{falls $n$ gerade ist}.
    \end{cases}
   \]
 \end{enumerate}
\end{solution}


\begin{question}
 Eine Matrix $A \in \Mat_n(k)$ heißt \emph{nilpotent} falls es ein $n \geq 1$ mit $A^n = 0$ gibt.
 \begin{enumerate}
  \item
   Zeigen Sie, dass die folgenden Matrizen nilpotent sind:
   \[
    \begin{pmatrix}
     0 & 1 \\
     0 & 0
    \end{pmatrix},
    \begin{pmatrix}
     0 & 0 \\
     1 & 0 \\
    \end{pmatrix},
    \begin{pmatrix}
     0 & 1 & 0 \\
     0 & 0 & 1 \\
     0 & 0 & 0
    \end{pmatrix},
    \begin{pmatrix}
     0 & 0 & 0 \\
     1 & 0 & 0 \\
     0 & 1 & 0
    \end{pmatrix}.
   \]
  \item
   Zeigen Sie: Ist $\lambda \in k$ und $A \in \Mat_n(k)$ nilpotent, so ist auch $\lambda A$ nilpotent.
  \item
   Finden Sie zwei nilpotente Matrizen $A, B \in \Mat_2(k)$, so dass $A+B$ nicht nilpotent ist.
  \item
   Zeigen Sie: Sind $A,B \in \Mat_n(k)$ zwei nilpotente, kommutierende Matrizen (d.h.\ $AB = BA$), so ist auch $A+B$ nilpotent. (\emph{Hinweis}: Finden Sie einen Ausdruck für $(A+B)^n$.)
  \item
   Zeigen Sie: Ist $A \in \Mat_n(k)$ nilpotent, so ist $I_n - A$ invertierbar. (\emph{Hinweis}: Falls $A^2 = 0$ betrachten Sie $I_n + A$. Falls $A^3 = 0$ betrachten Sie $I_n + A + A^2$.)
 \end{enumerate}
\end{question}
\begin{solution}
 \begin{enumerate}
  \item
   Durch Nachrechnen ergibt sich, dass
   \[
    \begin{pmatrix}
     0 & 1 \\
     0 & 0
    \end{pmatrix}^2
    = 0, 
    \begin{pmatrix}
     0 & 0 \\
     1 & 0
    \end{pmatrix}^2
    = 0,
    \begin{pmatrix}
     0 & 1 & 0 \\
     0 & 0 & 1 \\
     0 & 0 & 0
    \end{pmatrix}^3
    =0
    \quad \text{und} \quad
    \begin{pmatrix}
     0 & 0 & 0 \\
     1 & 0 & 0 \\
     0 & 1 & 0
    \end{pmatrix}^3
    = 0.
   \]
  \item
   Für alle $B_1, B_2 \in \Mat_n(k)$ und $\mu_1, \mu_2 \in k$ ist
   \[
    (\mu_1 B_1) (\mu_2 B_2) = \mu_1 \mu_2 (B_1 B_2).
   \]
   Insbesondere ist deshalb
   \[
    (\lambda A)^n = \lambda^n A^n \quad \text{für alle $n \geq 1$}.
   \]
   Da $A$ nilpotent ist gibt es $n \geq 1$ mit $A^n = 0$. Dann ist
   \[
    (\lambda A)^n = \lambda^n A^n = \lambda^n 0 = 0.
   \]
   Also ist auch $\lambda A$ nilpotent.
  \item
   Die Matrizen
   \[
    A =
    \begin{pmatrix}
     0 & 1 \\
     0 & 0
    \end{pmatrix}
    \quad \text{und} \quad
    \begin{pmatrix}
     0 & 0 \\
     1 & 0
    \end{pmatrix}
   \]
   sind nilpotent; es sei
   \[
    C \coloneqq A + B =
    \begin{pmatrix}
     0 & 1 \\
     1 & 0
    \end{pmatrix}.
   \]
   Durch Nachrechnen ergibt sich, dass $C^2 = I_2$. Es folgt, dass für alle $n \geq 1$
   \[
    C^n =
    \begin{cases}
     C   & \text{falls $n$ ungerade ist}, \\
     I_2 & \text{falls $n$ gerade ist}.
    \end{cases}
   \]
   Es gibt also kein $n \geq 1$ mit $C^n = 0$. Also ist $C$ nicht nilpotent.
  \item
   Da $A$ und $B$ nilpotent sind gibt es $n_A, n_B \geq 1$ mit $A^{n_A} = 0$ und $B^{n_B} = 0$. Es sei $n \coloneqq n_A + n_B$. Da $A$ und $B$ kommutieren gilt der Binomische Lehrsatz
   \[
    (A+B)^n = \sum_{\ell=0}^n \binom{n}{\ell} A^\ell B^{n-\ell}.
   \]
   Es ist $\ell \geq n_A$ oder $n-\ell \geq n_B$ für alle $\ell = 0, \dotsc, n$ da $n = n_A + n_B$. Also ist $A^\ell = 0$ oder $B^{n-\ell} = 0$ für alle $\ell = 0, \dotsc, n$. Es sind also alle Summanden $0$ und somit $(A+B)^n = 0$.
  \item
   Es sei $m \geq 1$ mit $A^m = 0$. Dann ist
   \begin{align*}
     &\, (I_n + A + A^2 + \dotsb + A^{m-1}) (I_n - A) \\
    =&\, (I_n + A + A^2 + \dotsb + A^{m-1}) - (A + A^2 + A^3 + \dotsb + A^m) \\
    =&\, I_n - A^m
    = I_n.
   \end{align*}
   Also ist $I_n - A$ invertierbar mit $(I_n - A)^{-1} = I_n + A + A^2 + \dotsb + A^{m-1}$.
 \end{enumerate}
\end{solution}


\begin{question}
 Es sei $G \subseteq \Mat_n(k)$ eine Teilmenge, die bezüglich der Matrizenmultiplikation eine Gruppe ist. Zeigen oder widerlegen Sie: $G$ besteht aus invertierbaren Matrizen und $G \subseteq \GL_n(k)$ ist eine Untergruppe.
\end{question}
\begin{solution}
 Die Aussage gilt nicht: Es sei $A \in \Mat_n(k)$ \emph{idempotent}, d.h.\ es ist $A^2 = A$. Dann ist $G \coloneqq \{A\} \subseteq \GL_n(k)$ eine Teilmenge, die bezüglich der Matrizenmultiplikation eine Gruppe bildet. Aber $A$ ist im Allgemeinen nicht invertierbar, siehe etwa
 \[
  A = \begin{pmatrix} 1 & 0 \\ 0 & 0 \end{pmatrix} \in \Mat_2(k).
 \]
 und die Nullmatrix $0 \in \Mat_n(k)$.
\end{solution}













