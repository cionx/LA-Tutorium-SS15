\section{Darstellende Matrizen und Basiswechsel}


\begin{question}
 Es sei
 \[
  A =
  \begin{pmatrix*}[r]
    2 & -3 \\
   -1 &  4
  \end{pmatrix*}
  \in \Mat_2(\R).
 \]
 Bestimmen Sie die darstellende Matrix der linearen Abbildung
 \[
  F \colon \Mat_2(\R) \to \Mat_2(\R), B \mapsto A \cdot B
 \]
 bezüglich der Basis $\mc{B} \coloneqq (E_{11}, E_{12}, E_{21}, E_{22})$ von $\Mat_2(\R)$ mit den Matrizen
 \[
  E_{11} = \begin{pmatrix} 1 & 0 \\ 0 & 0 \end{pmatrix},\;
  E_{12} = \begin{pmatrix} 0 & 1 \\ 0 & 0 \end{pmatrix},\;
  E_{21} = \begin{pmatrix} 0 & 0 \\ 1 & 0 \end{pmatrix},\;
  E_{22} = \begin{pmatrix} 0 & 0 \\ 0 & 1 \end{pmatrix}.
 \]
\end{question}
\begin{solution}
 Es ist
 \begin{align*}
  F(E_{11})
  &= A \cdot E_{11}
  = \begin{pmatrix*}[r] 2 & -3 \\ -1 & 4 \end{pmatrix*} \cdot \begin{pmatrix} 1 & 0 \\ 0 & 0 \end{pmatrix}
  = \begin{pmatrix*}[r] 2 & 0 \\ -1 & 0 \end{pmatrix*} = 2 E_{11} - E_{21}, \\
  F(E_{12})
  &= A \cdot E_{12}
  = \begin{pmatrix*}[r] 2 & -3 \\ -1 & 4 \end{pmatrix*} \cdot \begin{pmatrix} 0 & 1 \\ 0 & 0 \end{pmatrix}
  = \begin{pmatrix*}[r] 0 & 2 \\ 0 & -1 \end{pmatrix*} = 2 E_{12} - E_{22}, \\
  F(E_{21})
  &= A \cdot E_{21}
  = \begin{pmatrix*}[r] 2 & -3 \\ -1 & 4 \end{pmatrix*} \cdot \begin{pmatrix} 0 & 0 \\ 1 & 0 \end{pmatrix}
  = \begin{pmatrix*}[r] -3 & 0 \\ 4 & 0 \end{pmatrix*} = -3 E_{11} + 4E_{21}, \\
  F(E_{22})
  &= A \cdot E_{22}
  = \begin{pmatrix*}[r] 2 & -3 \\ -1 & 4 \end{pmatrix*} \cdot \begin{pmatrix} 0 & 0 \\ 0 & 1 \end{pmatrix}
  = \begin{pmatrix*}[r] 0 & -3 \\ 0 & 4 \end{pmatrix*} = -3 E_{12} + 4E_{22}.
 \end{align*}
 Also ist
 \[
  \DM_\mc{B}(F) =
  \begin{pmatrix}
    2 &  0 & -3 &  0 \\
    0 &  2 &  0 & -3 \\
   -1 &  0 &  4 &  0 \\
    0 & -1 &  0 &  4
  \end{pmatrix}.
 \]
\end{solution}


\begin{question}
 Es sei $n \geq 1$ und $V_n \subseteq \R[X]$ der Untervektorraum der reellen Polynome vom \mbox{Grad $\leq n$}. Es sei
 \[
  D \colon V_n \to V_n, \, p \mapsto p'.
 \]
 Bestimmen Sie die darstellende Matrix von $D$ bezüglich der Basis
 \[
  \mc{B} \coloneqq (1, x, x^2, \dotsc, x^n)
 \]
 von $V_n$.
\end{question}
\begin{solution}
 Es ist
 \[
  D(1) = 0, \quad D(x) = 1, \quad D(x^2) = 2x, \quad \dotsc, \quad D(x^n) = n x^{n-1}.
 \]
 Also ist
 \[
  \DM_\mc{\mc{B},\mc{B}}(D) =
  \begin{pmatrix}
   0 & 1 &        &        &   \\
     & 0 & 2      &        &   \\
     &   & \ddots & \ddots &   \\
     &   &        & 0      & n \\
     &   &        &        & 0
  \end{pmatrix}.
 \]
\end{solution}


\begin{question}
 Es sei $\mc{B} \coloneqq (b_1, b_2, b_3)$ die Basis von $\R^3$ mit
 \[
  b_1 = \vect{1 \\ 0 \\ 0}, \quad
  b_2 = \vect{1 \\ 1 \\ 0}, \quad
  \text{und} \quad
  b_3 = \vect{1 \\ 1 \\ 1}.
 \]
 Finden Sie eine Matrix $A \in \Mat_3(\R)$ mit
 \[
  A b_1 = b_2, \quad
  A b_2 = b_3, \quad
  \text{und} \quad
  A b_3 = b_1.
 \]
\end{question}
\begin{solution}
 Es sei $A \in \Mat_3(\R)$ und $a_i$ der $i$-te Spaltenvektor von $A$ für $i = 1, 2, 3$. Dann ist
 \[
  A b_1 = a_1, \quad
  A b_2 = a_1 + a_2, \quad
  \text{und} \quad
  A b_3 = a_1 + a_2 + a_3.
 \]
 Das $A b_1 = b_2$ ist also äquivalent dazu, dass $a_1 = b_2$. Dass  $A b_2 = b_3$ ist damit äquivalent dazu, dass \[
  b_3 = a_1 + a_2 = b_2 + a_2,
 \]
 also $a_2 = b_3 - b_2$. Dass $A b_3 = b_1$ ist damit äquivalent dazu, dass
 \[
  b_1 = a_1 + a_2 + a_3 = b_2 + b_3 - b_2 + a_3 = b_3 + a_3,
 \]
 also $a_3 =  b_1 - b_3$. Da
 \[
  a_1 = b_2 = \vect{1 \\ 1 \\ 0}, \quad
  a_2 = b_3 - b_2 = \vect{0 \\ 0 \\ 1}, \quad
  \text{und} \quad
  a_3 = b_1 - b_3 = \vect{0 \\ -1 \\ -1}
 \]
 ist
 \[
  A =
  \begin{pmatrix}
   1 & 0 &  0 \\
   1 & 0 & -1 \\
   0 & 1 & -1
  \end{pmatrix}.
 \]
 Durch direktes Nachrechnen ergibt sich, dass $A$ die gewünschten Bedingungen tatsächlich erfüllt.
 
 Alternativ sei $f \colon \R^3 \to \R^3$ die eindeutige lineare Abbildung mit $f(b_1) = b_2$, $f(b_2) = b_3$ und $f(b_3) = b_1$. ($f$ existiert und ist eindeutig, da $\mc{B}$ eine Basis von $\R^3$ ist.) Dann ist
 \[
  \DM_{\mc{B},\mc{B}}(f) =
  \begin{pmatrix}
   0 & 0 & 1 \\
   1 & 0 & 0 \\
   0 & 1 & 0
  \end{pmatrix}
 \]
 und die gesuchte Matrix $A$ ist
 \[
  A = \DM_{\mc{K},\mc{K}}(f),
 \]
 wobei $\mc{K} = (e_1, e_2, e_3)$ die Standardbasis von $\R^3$ ist. Da
 \[
  T^\mc{B}_\mc{K} =
  \begin{pmatrix}
   1 & 1 & 1 \\
   0 & 1 & 1 \\
   0 & 0 & 1
  \end{pmatrix}
  \quad\text{und damit}\quad
  T^\mc{K}_\mc{B} =
  (T^\mc{B}_\mc{K})^{-1}
  \begin{pmatrix}
   1 & -1 &  0 \\
   0 &  1 & -1 \\
   0 &  0 &  1
  \end{pmatrix}
 \]
 ist
 \begin{align*}
  A
  &= \DM_{\mc{K},\mc{K}}(f)
  = T^\mc{B}_\mc{K} \DM_{\mc{B},\mc{B}}(f) T^\mc{K}_\mc{B} \\
  &=
  \begin{pmatrix}
   1 & 1 & 1 \\
   0 & 1 & 1 \\
   0 & 0 & 1
  \end{pmatrix}
  \begin{pmatrix}
   0 & 0 & 1 \\
   1 & 0 & 0 \\
   0 & 1 & 0
  \end{pmatrix}
  \begin{pmatrix}
   1 & -1 &  0 \\
   0 &  1 & -1 \\
   0 &  0 &  1
  \end{pmatrix}
  =
  \begin{pmatrix}
   1 & 0 &  0 \\
   1 & 0 & -1 \\
   0 & 1 & -1
  \end{pmatrix}.
 \end{align*}
\end{solution}




\begin{question}
 Für $z \in \C$ sei
 \[
  L_z \colon \C \to \C, w \mapsto z w
 \]
 die Multiplikation mit $z$.
 \begin{enumerate}
  \item
   Zeigen Sie für alle $z \in \C$, dass $L_z$ eine $\R$-lineare Abbildung ist.
  \item
   Es sei $z \in \C$. Geben Sie die darstellende Matrix von $L_z$ bezüglich der $\R$-Basis $\mc{B} \coloneqq (1,i)$ von $\C$ an, d.h.\ $\DM_\mc{B}(L_z)$.
  \item
   Zeigen Sie, dass die Abbildung $\Phi \colon \C \to \Mat_2(\R), z \mapsto \DM_\mc{B}(L_z)$ injektiv ist.
  \item
   Zeigen Sie, dass $\Phi(1) = I_2$ und dass für alle $z, w \in \C$
   \[
    \Phi(z+w) = \Phi(z) + \Phi(w)
    \quad \text{und} \quad
    \Phi(z \cdot w) = \Phi(z) \cdot \Phi(w).
   \]
 \end{enumerate}
\end{question}
\begin{solution}
 \begin{enumerate}
  \item
   Für alle $w, w_1, w_2 \in \C$ und $\lambda \in \C$ ist
   \[
    L_z(\lambda w) = z \lambda w = \lambda z w = \lambda L_z(w)
   \]
   und
   \[
    L_z(w_1 + w_2) = z(w_1 + w_2) = z w_1 + z w_2 = L_z(w_1) + L_z(w_2),
   \]
   also ist $L_z$ eine $\C$-lineare Abbildung. Insbesondere ist $L_z$ deshalb auch $\R$-linear.
  \item
   Es sei $a \coloneqq \Re(z)$ und $b \coloneqq \Im(z)$, also $z = a + bi$ mit $a, b \in \R$. Dann ist
   \begin{gather*}
    L_z(1) = z \cdot 1 = z = a + b i
   \shortintertext{und}
    L_z(i) = z \cdot i = (a + i b) \cdot i = -b + i a.
   \end{gather*}
   Die darstellende Matrix von $L_z$ bezüglich der Basis $\mc{B}$ ist deshalb
   \[
    \DM_\mc{B}(L_z) =
    \begin{pmatrix}
     a & -b \\
     b &  a \\
    \end{pmatrix}.
   \]
  \item
   Es seien $z, z' \in \R$ mit $z = a + i b$ und $z' = a' + i b'$ mit $a, b, a', b' \in \R$. Ist $z \neq z'$, so ist $a \neq a'$ oder $b \neq b'$. Dann sind auch
   \[
    \DM_\mc{B}(L_z) =
    \begin{pmatrix}
     a & -b \\
     b &  a \\
    \end{pmatrix}
    \quad\text{und}\quad
    \DM_\mc{B}(L_{z'}) =
    \begin{pmatrix}
     a' & -b' \\
     b' &  a' \\
    \end{pmatrix}
   \]
   verschieden. Also ist $\Phi$ injektiv.
   
   Die Aussage lässt sich auch konzeptioneller beweisen: $\Phi$ lässt sich als Komposition
   \[
    \Phi \colon
    \C
    \xlongrightarrow[z \mapsto L_z]{\Phi_1}
    \Hom_\R(\C,\C)
    \xlongrightarrow[f \mapsto \DM_\mc{B}(f)]{\Phi_2}
    \Mat_2(\R)
   \]
   schreiben. Die Abbildung $\Phi_1 \colon \C \to \C, z \mapsto L_z$ ist injektiv, denn für $z, z' \in \C$ mit $L_z = L_{z'}$ ist
   \[
    z = L_z(1) = L_{z'}(1) = z'.
   \]
   Von der Abbildung $\Hom_\R(\C,\C) \to \Mat_2(\R)$, $f \mapsto \DM_\mc{B}(f)$ ist aus der Vorlesung bekannt, dass sie eine Bijektion ist (und sogar ein Isomorphismus von $\R$-Vektorräumen). Damit ist $\Phi$ als Verknüpfung zweier injektiver Funktionen ebenfalls injektiv.
  \item
   Die Aussage lässt sich durch direktes Nachrechnen zeigen: Für $z, w \in \C$ mit $z = a + i b$ und $w = u + i v$ mit $a,b,u,v \in \R$ ist $z + w = (a + u) + i (b + w)$ und daher
   \begin{align*}
    \Phi(z) + \Phi(w)
    &= \DM_\mc{B}(L_z) + \DM_\mc{B}(L_w)
    = \begin{pmatrix} a & -b \\ b & a \end{pmatrix} + \begin{pmatrix} u & -v \\ v & u \end{pmatrix} \\
    &= \begin{pmatrix} a + u & -(b + v) \\ b + v & a + u \end{pmatrix} 
    = \DM_\mc{B}(L_{z+w})
    = \Phi(z+w).
   \end{align*}
   Außerdem ist $z \cdot w = (a u - b v) + i (a v + b u)$ und deshalb
   \begin{align*}
    \Phi(z) \cdot \Phi(w)
    &= \DM_\mc{B}(L_z) \cdot \DM_\mc{B}(L_w)
    = \begin{pmatrix} a & -b \\ b & a \end{pmatrix} \cdot \begin{pmatrix} u & -v \\ v & u \end{pmatrix} \\
    &= \begin{pmatrix} a u - b v & - (a v + b u) \\ a v + b u & a u - b v \end{pmatrix} 
    = \DM_\mc{B}(L_{z+w})
    = \Phi(z \cdot w).
   \end{align*}
   
   Die Aussage lässt sich auch konzeptioneller zeigen: Für $z, w \in \C$ ist $L_{z + w} = L_z + L_w$, da für alle $\xi \in \C$
   \[
    L_{z+w}(\xi) = (z+w) \xi = z \xi + w \xi = L_z(\xi) + L_w(\xi) = (L_z + L_w)(\xi)
   \]
   und $L_{zw} = L_z \circ L_w$, da für alle $\xi \in \C$
   \[
    L_{zw}(\xi) = z w \xi = L_z(w \xi) = L_z( L_w(\xi) ) = (L_z \circ L_w)(\xi).
   \]
   Aus der Vorlesung ist auch bekannt, dass für alle $f, g \in \Hom_\R(\C,\C)$
   \[
    \DM_\mc{B}(f + g) = \DM_\mc{B}(f) + \DM_\mc{B}(g)
    \quad\text{und}\quad
    \DM_\mc{B}(f \circ g) = \DM_\mc{B}(f) \cdot \DM_\mc{B}(g).
   \]
   Für alle $z, w \in \C$ ist dehalb
   \begin{align*}
    \Phi(z+w)
    &= \DM_\mc{B}(L_{z+w})
    = \DM_\mc{B}(L_z + L_w) \\
    &= \DM_\mc{B}(L_z) + \DM_\mc{B}(L_w)
    = \Phi(z) + \Phi(w)
   \shortintertext{sowie}
    \Phi(z \cdot w)
    &= \DM_\mc{B}(L_{z \cdot w})
    = \DM_\mc{B}(L_z \circ L_w) \\
    &= \DM_\mc{B}(L_z) \cdot \DM_\mc{B}(L_w)
    = \Phi(z) \cdot \Phi(w).
   \end{align*}
 \end{enumerate}
\end{solution}

